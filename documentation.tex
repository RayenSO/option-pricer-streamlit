\documentclass{article}
\usepackage{amsmath}
\usepackage{amssymb}
\usepackage{geometry}
\usepackage{graphicx}
\usepackage{hyperref}
\usepackage{booktabs}

\geometry{a4paper, margin=1in}

\title{Documentation Projet: Pricing d'Options Financières}
\author{SOLTANI Rayen, ELIASY Ny Avotra}
\date{\today}

\begin{document}

\maketitle

\tableofcontents
\newpage

\section{Introduction}

Ce projet a pour objectif de mettre en application les notions théoriques étudiées dans le cour de "Applied Mathematics for Finance" du Master 2 Gestion de Portefeuille de l'IAE Paris-Est dispensé par Monsieur Al-Wakil.

Nous avons choisi de travailler sur les trois modèles vu en cours:
\begin{itemize}
    \item Le modèle de \textbf{Black-Scholes} (option européenne uniquement)
    \item L'\textbf{Arbre Binomial de Cox-Ross-Rubinstein} (options européennes et américaines)
    \item La \textbf{Méthode de Monte Carlo} (option européenne)
\end{itemize}

\section{Modèle de Black-Scholes}
\subsection{Principe général}

Le modèle de Black-Scholes repose sur une hypothèse de dynamique log-normale du prix du sous-jacent et permet de dériver une formule analytique fermée pour le prix des options européennes, sous certaines conditions de marché idéalisées (pas de dividende, marchés sans frais de transaction, une liquidité parfaite, volatilité et taux constants).

\subsection{Formule générale}

Pour une \textbf{option Call européenne}, le prix théorique \( C \) est donné par :

\begin{equation}
C = S_0 N(d_1) - Ke^{-rT} N(d_2)
\end{equation}

Pour une \textbf{option Put européenne} :

\begin{equation}
P = Ke^{-rT} N(-d_2) - S_0 N(-d_1)
\end{equation}

où :

\begin{align}
d_1 &= \frac{\ln(S_0/K) + (r + \frac{1}{2}\sigma^2)T}{\sigma \sqrt{T}} \\
d_2 &= d_1 - \sigma \sqrt{T}
\end{align}

\subsection{Interprétation des paramètres}

\begin{itemize}
    \item \( S_0 \) : Prix du sous-jacent aujourd'hui
    \item \( K \) : Prix d'exercice (strike)
    \item \( T \) : Temps à maturité (en années)
    \item \( r \) : Taux d'intérêt sans risque (continu)
    \item \( \sigma \) : Volatilité du sous-jacent
    \item \( N(\cdot) \) : Fonction de répartition de la loi normale standard
\end{itemize}

\section{Arbre Binomial de Cox-Ross-Rubinstein (CRR)}

\subsection{Structure de l'arbre}

L'arbre binomial modélise le prix du sous-jacent par une série de montées et descentes possibles à chaque pas de temps.

\begin{itemize}
    \item Facteur de hausse : \( u = e^{\sigma \sqrt{\Delta t}} \)
    \item Facteur de baisse : \( d = e^{-\sigma \sqrt{\Delta t}} = 1/u \)
    \item Probabilité neutre au risque : 
    \begin{equation}
    p = \frac{e^{r\Delta t} - d}{u - d}
    \end{equation}
\end{itemize}

\subsection{Calcul du prix}

Le prix de l'option est obtenu par :

\begin{enumerate}
    \item Construction de l'arbre des prix de l'actif
    \item Calcul des payoffs aux nœuds terminaux
    \item Remontée de l'arbre par actualisation des valeurs :
    \begin{equation}
    \text{Valeur du nœud} = e^{-r \Delta t} (p \times \text{Valeur du nœud haut} + (1-p) \times \text{Valeur du nœud bas})
    \end{equation}
\end{enumerate}

Pour une option américaine, on prend à chaque nœud :

\[
\text{Valeur} = \max(\text{Payoff immédiat}, \text{Valeur de continuation})
\]

\section{Méthode de Monte Carlo}

\subsection{Principe général}

La méthode Monte Carlo simule de nombreux chemins possibles pour le prix du sous-jacent à l'échéance selon :

\begin{equation}
S_T = S_0 \exp\left( \left(r - \frac{1}{2}\sigma^2\right)T + \sigma \sqrt{T} Z \right)
\end{equation}

où \( Z \sim \mathcal{N}(0,1) \) est un tirage aléatoire de loi normale standard.

\subsection{Prix de l'option}

Le prix de l'option est donné par :

\begin{equation}
\text{Prix} = e^{-rT} \mathbb{E}[\text{Payoff}]
\end{equation}

\[
\text{Payoff Call} = \max(S_T - K, 0)
\quad\quad
\text{Payoff Put} = \max(K - S_T, 0)
\]

La moyenne est prise sur toutes les simulations.

\section{Les Grecques (Greeks)}

\subsection{Introduction}

Les \textbf{Greeks} mesurent la sensibilité du prix de l'option à différentes variables de marché.

\subsection{Définitions et Formules}

\begin{itemize}
    \item \textbf{Delta} : Sensibilité du prix de l'option au prix du sous-jacent.
    \[
    \Delta = \frac{\partial C}{\partial S_0}
    \]

    \item \textbf{Gamma} : Sensibilité du Delta au prix du sous-jacent.
    \[
    \Gamma = \frac{\partial^2 C}{\partial S_0^2}
    \]

    \item \textbf{Vega} : Sensibilité du prix de l'option à la volatilité du sous-jacent.
    \[
    \text{Vega} = \frac{\partial C}{\partial \sigma}
    \]

    \item \textbf{Theta} : Sensibilité du prix de l'option au passage du temps.
    \[
    \Theta = \frac{\partial C}{\partial T}
    \]

    \item \textbf{Rho} : Sensibilité du prix de l'option au taux sans risque.
    \[
    \text{Rho} = \frac{\partial C}{\partial r}
    \]
\end{itemize}

\subsection{Interprétation intuitive}

\begin{itemize}
    \item \textbf{Delta} : Variation approximative du prix de l'option pour une variation de 1 unité de prix du sous-jacent.
    \item \textbf{Gamma} : Mesure la convexité ; variation du Delta.
    \item \textbf{Vega} : Variation du prix de l'option pour une variation de 1\% de la volatilité.
    \item \textbf{Theta} : Perte de valeur de l'option par jour qui passe.
    \item \textbf{Rho} : Variation du prix de l'option pour une variation de 1\% du taux sans risque.
\end{itemize}


\end{document}
